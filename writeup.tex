% {{{
\input{$HOME/.config/nvim/snippets/.math.preamble.tex}
% }}}

% Title {{{
\begin{document}

\begin{center}
	{\large \bf Jason Abaluck }   \\ \large optimal Representation \\ Ephraim Sutherland
\end{center}
% }}}

\subsection*{Setup}

\begin{enumerate}

	\item  Suppose a physician can only see see ATE and some measure of representativeness. They have prior
		$\bar{\beta}$ and  $\beta_{ATE} = (1/N) \sum \beta_{i}$.

	\item need model for betas related to each other based on $x$'s.
		WLOG, suppose
		\begin{align*}
			\beta(x_i) = x_i \gamma
		\end{align*} 
		Where $x_i$ is a vector of characteristics and $\gamma$ is a vector of coefficients. \\
		If you know $\gamma$, then you know $\beta$ for any given patient.
	\item However, you don't observe $\gamma$, you instead observe:
		$\beta_{ATE} = \bar x \gamma$ where $\bar x = (\frac{1}{N}) \sum x_i$
	\item We know $\beta_i$ for patients with characteristics $\bar x$ (it is $\beta_{ATE}$).
	\item For other patients, need to solve
		\begin{align*}
			\beta_{i,post} = \E(x_i \gamma | \bar x \gamma = \beta_{ATE})
		\end{align*} 

	\item to solve
		\begin{enumerate}
			\item
				\begin{align*}
					\beta_{i,post} & = \E (x_i \gamma | \bar x \gamma = \beta_{ATE})                                                                        \\
								   & = \E((x_i - c_i \bar{x}) \gamma | \bar{x} \gamma = \beta_{ATE}) + c_i \E(\bar{x} \gamma | \bar{x} \gamma = \beta_{ATE}) \\
								   & = \E((x_{i} - c_i \bar{x}) \gamma | \bar{x} \gamma = \beta_{ATE}) + c_i \beta_{ATE}                                       \\
				\end{align*}
				For any constant $c_i$. \\
				Choose $c_i$ so that 
				\begin{align*}
					\Cov((x_{i} - c_i \bar{x})\gamma, \bar{x} \gamma) = 0
				\end{align*} 
				maybe assume normality so that this guarantees independence.
				Then,
				\begin{align*}
					\E((x_i - c_i \bar{x}) \gamma | \bar{x} \gamma = \beta_{ATE}) = (x_{i} - c_i \bar{x}) \E(\gamma)
				\end{align*}
				So then
				\begin{align*}
					(x_i - c_i \bar{x}) \E(\gamma) + c_i \beta_{ATE} = x_i \E(\gamma) + c_i (\beta_{ATE} - \bar{x} \E(\gamma))
				\end{align*} 
				($c_i$ depends on $x_i$)

				In other words, your belief is your prior, adjusted based on the difference between the observed ATE and your prior about the ATE.
				The key question is how much adjustment you do which depends on $"c_i"$. We choose $c_i$ to solve:

				\begin{align*}
                          & \Cov ((x_i - c_i \bar{x}) \gamma, \bar{x} \gamma) = 0                          \\
				 \iff     & \Cov(x_i \gamma, \bar{x} \gamma) -c_i \Cov(\bar{x} \gamma, \bar{x} \gamma) = 0 \\
					\iff  & \Cov(x_i \gamma, \bar{x} \gamma) = c_i \Var(\bar{x} \gamma)                  \\
				\iff & c_i = \frac{\Cov(\beta_i, \beta_{ATE}) }{ \Var(\beta_{ATE})}
				\end{align*} 
				The random variable in this context is $\gamma$ (the coefficients on the $x $'s) in this case $\Var(\beta_{ATE})$ is a measure of how uncertain one was about what $\beta_{ATE}$ would be before doing the trial.

				$c_i$ is the equation for a regression of $\beta_i$ on $\beta_{ATE}$. In other words, we take a bunch of patients with characteristics $x_i$ and we keep redrawing the gammas from our prior distribution the we ask how correlated  $\beta_i$ and $\beta_{ATE}$ are. If they are more correlated (as they would be for patients where the $x_i$ are closer to $\bar{x}$ we update more.

				To compute $c_i$, we just need to know $x_i$ $\bar{x}$, and the distribution of $\gamma$.

				Suppose we want to design the trial to minimize:
				\begin{align*}
					\min \E[(\beta_i - \beta_{i,post})^2]
				\end{align*} 
		\end{enumerate}

\end{enumerate} 

\subsection*{Simple Cases}
\begin{enumerate}
	\item There is just one $x$ and it is binary (old v young). Can it be solved analytically?
	\item Can you solve a 2-dimensional case? 
\end{enumerate}


\subsection*{Solving}

we want to solve
\begin{align*}
	\min \E[(\beta_i - \beta_{i,post})^2]
\end{align*} 

let's start by considering
\begin{align*}
	\E[(\beta_i - \beta_{i,post})^2] &= \E([ x \gamma  - ((x - c\bar{x}) \E(\gamma) + c \beta_ATE)]^2) \\
									 &= \E( [x \gamma - x \E(\gamma) + c \bar{x} \E(\gamma) - c \bar{x} \gamma ]^2 ) \\
									 &= \E ( [x ( \gamma - \E(\gamma)) - c\bar{x} ( \gamma - \E(\gamma))]^2) \\
									 &= \E([ ( x - c \bar{x}) (\gamma - \E(\gamma))]^2) \\
									 &= \E[ ( x - c \bar{x})^2 (\gamma - \E(\gamma))^2] \\
									 &= \E[ ( x^2 - 2 c \bar{x}x + (c \bar{x})^2) (\gamma - \E(\gamma))^2] \\
									 &= \E[ ( x^2 (\gamma - \E(\gamma))^2]  - \E[ ( 2 c \bar{x}x (\gamma - \E(\gamma))^2] + \E[ ( (c \bar{x})^2) \bar{x}x (\gamma - \E(\gamma))^2] \\
\end{align*} 

We must also solve $c$ and make a claim about the distribution of $x$.

We have that
\begin{align*}
	c = \frac{\Cov (x \gamma, \bar{x} \gamma)}{ \Var( \bar{x} \gamma )}
\end{align*} 

so if we have a single characteristic and no intercept, then for individual i, $c$ reduces to
\begin{align*}
	c &= \frac{\Cov (x \gamma, \bar{x} \gamma)}{ \Var( \bar{x} \gamma )} = \frac{x \bar{x} \Var(\gamma)}{\bar{x}^2 \Var(\gamma)} = \frac{x}{\bar{x}}
\end{align*} 


note that the above formula also captures if we added an intercept. In this case, if we assume $\beta_i = \gamma_1 + \gamma_2 x$ and both gammas are drawn from the same distribution and are independent, then the equation would then simply become
\begin{align*}
	c = \frac{(1^2 + x \bar{x})\Var(\gamma)}{(1^2 + \bar{x}^2)\Var(\gamma)} = \frac{1 + x \bar{x}}{1 + \bar{x}^2} = \frac{x' \bar{x}}{\bar{x}' \bar{x}}
\end{align*} 

% But if we include an intercept, then
% \begin{align*}
% 	c &= \frac{\Cov (x \gamma, \bar{x} \gamma)}{ \Var( \bar{x} \gamma )} \\
% 	  &= \frac{\Var(\gamma) + \bar{x} \cov}{}
% \end{align*} 

note, this is for individual $i$ so $x$ is given. Thus the only random variable in this context is $\gamma$

going back to the objective function, then now we are taking the expectation over all individuals, so we can look at the expected population characteristic as a bernoulli r.v. so
$\E(x^2) = \E(x) = p$ and we get
 \begin{align*}
	 &\E[ ( x^2 (\gamma - \E(\gamma))^2]  - \E[ ( 2 c \bar{x}x (\gamma - \E(\gamma))^2] + \E[ ( (c \bar{x})^2) \bar{x}x (\gamma - \E(\gamma))^2] \\
	 &= \Var(\gamma) [\E(x) - \E(2c\bar{x} x) + \E((c\bar{x})^2)]
\end{align*} 
In our case, we're assuming $x=[1, x]$ and  $\gamma = [\gamma_0, \gamma_1]$. Ignore dimensions for now. We can take care of proper transposes etc later to make this formally correct. But this implies we get $\Var(\gamma) = \Var(\gamma_0) + \Var(\gamma_1)$ (by assumption of being uncorrelated).
\begin{align*}
	  &\Var(\gamma) [\E(x) - \E(2c\bar{x} x) + \E((c\bar{x})^2)] \\
	= &\left( \Var(\gamma_0) + \Var(\gamma_1) \right)\left[1 + \E(x) - \frac{2\bar{x}}{1 + \bar{x}^2} \E(x + x^2 \bar{x}) + \frac{\bar{x}^2}{1 + \bar{x}^2} \E(1 + x \bar{x}) \right] \\
	= &\left( \Var(\gamma_0) + \Var(\gamma_1) \right)\left[1 + \E(x) - \frac{2\bar{x}}{1 + \bar{x}^2} \left(\E(x) +  \bar{x}\E(x)\right) + \frac{\bar{x}^2}{1 + \bar{x}^2} + \bar{x}\frac{\bar{x}^2}{1 + \bar{x}^2} \E(x) ) \right] \\
\end{align*} 
\end{document}


