% {{{
\input{$HOME/.config/nvim/snippets/.math.preamble.tex}
% }}}

% Title {{{
\begin{document}

\begin{center}
	{\large \bf Jason Abaluck }   \\ \large optimal Representation \\ Ephraim Sutherland
\end{center}
% }}}

\tableofcontents


% First we'll start by examing a simple scenario where two subgroups have equal variance (e.g. considermen and women).


\subsection*{Setup}


\begin{enumerate}

	\item  Suppose a physician can only see see ATE and some measure of representativeness. They have prior
		$\bar{\beta}$ and  $\beta_{ATE} = (1/N) \sum \beta_{i}$.

	\item need model for betas related to each other based on $x$'s.
		WLOG, suppose
		\begin{align*}
			\beta(x_i) = x_i \gamma
		\end{align*} 
		Where $x_i$ is a vector of characteristics and $\gamma$ is a vector of coefficients. \\
		If you know $\gamma$, then you know $\beta$ for any given patient.
	\item However, you don't observe $\gamma$, you instead observe:
		$\beta_{ATE} = \bar x \gamma$ where $\bar x = (\frac{1}{N}) \sum x_i$
	\item We know $\beta_i$ for patients with characteristics $\bar x$ (it is $\beta_{ATE}$).
	\item For other patients, need to solve
		\begin{align*}
			\beta_{i,post} = \E(x_i \gamma | \bar x \gamma = \beta_{ATE})
		\end{align*} 

	\item to solve
		\begin{enumerate}
			\item
				\begin{align*}
					\beta_{i,post} & = \E (x_i \gamma | \bar x \gamma = \beta_{ATE})                                                                        \\
								   & = \E((x_i - c_i \bar{x}) \gamma | \bar{x} \gamma = \beta_{ATE}) + c_i \E(\bar{x} \gamma | \bar{x} \gamma = \beta_{ATE}) \\
								   & = \E((x_{i} - c_i \bar{x}) \gamma | \bar{x} \gamma = \beta_{ATE}) + c_i \beta_{ATE}                                       \\
				\end{align*}
				For any constant $c_i$. \\
				Choose $c_i$ so that 
				\begin{align*}
					\Cov((x_{i} - c_i \bar{x})\gamma, \bar{x} \gamma) = 0
				\end{align*} 
				maybe assume normality so that this guarantees independence.
				Then,
				\begin{align*}
					\E((x_i - c_i \bar{x}) \gamma | \bar{x} \gamma = \beta_{ATE}) = (x_{i} - c_i \bar{x}) \E(\gamma)
				\end{align*}
				So then
				\begin{align*}
					(x_i - c_i \bar{x}) \E(\gamma) + c_i \beta_{ATE} = x_i \E(\gamma) + c_i (\beta_{ATE} - \bar{x} \E(\gamma))
				\end{align*} 
				($c_i$ depends on $x_i$)

				In other words, your belief is your prior, adjusted based on the difference between the observed ATE and your prior about the ATE.
				The key question is how much adjustment you do which depends on $"c_i"$. We choose $c_i$ to solve:

				\begin{align*}
                          & \Cov ((x_i - c_i \bar{x}) \gamma, \bar{x} \gamma) = 0                          \\
				 \iff     & \Cov(x_i \gamma, \bar{x} \gamma) -c_i \Cov(\bar{x} \gamma, \bar{x} \gamma) = 0 \\
					\iff  & \Cov(x_i \gamma, \bar{x} \gamma) = c_i \Var(\bar{x} \gamma)                  \\
				\iff & c_i = \frac{\Cov(\beta_i, \beta_{ATE}) }{ \Var(\beta_{ATE})}
				\end{align*} 
				The random variable in this context is $\gamma$ (the coefficients on the $x $'s) in this case $\Var(\beta_{ATE})$ is a measure of how uncertain one was about what $\beta_{ATE}$ would be before doing the trial.

				$c_i$ is the equation for a regression of $\beta_i$ on $\beta_{ATE}$. In other words, we take a bunch of patients with characteristics $x_i$ and we keep redrawing the gammas from our prior distribution the we ask how correlated  $\beta_i$ and $\beta_{ATE}$ are. If they are more correlated (as they would be for patients where the $x_i$ are closer to $\bar{x}$ we update more.

				To compute $c_i$, we just need to know $x_i$ $\bar{x}$, and the distribution of $\gamma$.

				Suppose we want to design the trial to minimize:
				\begin{align*}
					\min \E[(\beta_i - \beta_{i,post})^2]
				\end{align*} 
		\end{enumerate}

\end{enumerate} 

\subsection*{Simple Cases}
\begin{enumerate}
	\item There is just one $x$ and it is binary (old v young). Can it be solved analytically?
	\item Can you solve a 2-dimensional case? 
\end{enumerate}


First observe that in our current setup, we have that
\begin{align*}
	c = \frac{\Cov ( \gamma_0 + \gamma_1 x , \gamma_0  + \gamma_1 \bar{x} )}{ \Var( \gamma_0  + \gamma_1 \bar{x} ))}
\end{align*} 


so for individual i, $c$ reduces to

\begin{align*}
		c &= \frac{\Var(\gamma_0) + (x + \bar{x}) \Cov(\gamma_0, \gamma_1) + x \bar{x} \Var(\gamma_1)}{\Var(\gamma_0) + 2 \bar{x} \Cov(\gamma_0, \gamma_1) + \bar{x}^2\Var(\gamma_1)} \\
		C_{women} &= \frac{\Var(\gamma_0) + \bar{x} \Cov(\gamma_0, \gamma_1)}{\Var(\gamma_0) + 2 \bar{x} \Cov(\gamma_0, \gamma_1) + \bar{x}^2\Var(\gamma_1)} \\
	C_{men} &= \frac{\Var(\gamma_0) + (1 + \bar{x}) \Cov(\gamma_0, \gamma_1) + \bar{x} \Var(\gamma_1)}{\Var(\gamma_0) + 2 \bar{x} \Cov(\gamma_0, \gamma_1) + \bar{x}^2\Var(\gamma_1)} \\
\end{align*} 


\section{Results}


Note, we use $p$ to be the population proportion and $\bar{x}$ to be the average characteristics in the clinical trial. For simplicity, we will call $x = 1$ men, and $x = 0$ women and thus $\bar{x}$ represents the proportion of men in the trial.

\subsection*{Case 1}
For Case 1, we have 

\begin{align*}
	\beta_i &= (1 - x) \gamma_0 + x \gamma_1 \\
	\beta_{ATE} &= (1 - \bar{x}) \gamma_0 + \bar{x} \gamma_1 \\
\end{align*}

For this case we get that whenever $p > \frac{\gamma_0}{ \gamma_0 + \gamma_1}$. When this is true, we only choose men to be in our trial. If $p < \frac{\gamma_0}{ \gamma_0 + \gamma_1}$ we choose all women.
Otherwise, if $p = \frac{\gamma_0}{ \gamma_0 + \gamma_1}$ it doesn't matter what proportions we include in our trial, the MSE will be the same.

\subsection*{Case 2}

For Case 2, we have 

\begin{align*}
	\beta_i &=  \gamma_0 + x \gamma_1 \\
	\beta_{ATE} &=  \gamma_0 + \bar{x} \gamma_1 \\
\end{align*}


For Case $2$, we see that we always have a preference for more men (if we have a population proportion of $p = \frac{1}{2}$, then we would want $\approx 61.8 \%$ of the trial to be men.
This is because they have a higher variance in outcomes. To further investigate the effect of variance in this case, can also observe that as $\Var(\gamma_1) \to \infty$ holding $\Var(\gamma_0)$ fixed we always choose only men.
And as $\Var(\gamma_0) \to \infty$ holding $\Var(\gamma_1)$ fixed, we choose $\bar{x} = p$, equal to the population proportion.

\subsection*{Case 3}

In this case, we consider case 2
\begin{align*}
	\beta_i &=  \gamma_0 + x \gamma_1 \\
	\beta_{ATE} &=  \gamma_0 + \bar{x} \gamma_1 \\
\end{align*}

but allowing for arbitrary correlation betweem $\gamma_0$ and $\gamma_1$.

First, we can see that when we let  $\Cov(\gamma_0, \gamma_1) = 0$, then we recover case 2.
% But if we allow $\Cov(\gamma_0, \gamma_1) = - \Var(\gamma_0)$

Furthermore, our intuition from the previous results suggest that the representation depends on the respective variances in outcomes of each group.

We will thus consider a set of options.

Given the setup, we can observe that
\begin{align*}
	\Var(\beta_{women}) &= \Var(\gamma_0) \\
	\Var(\beta_{men}) &= \Var(\gamma_0) + 2\Cov(\gamma_0, \gamma_1) + \Var(\gamma_1) \\
\end{align*} 

So we will impose that $\Var(\gamma_0) = \Var(\gamma_1) = 1$ and consider different covariances to make men or women have more variance.

From the variances for outcomes above, we can see that for men and women to have equal variance we must impose that  $\Cov(\gamma_0, \gamma_1) = -\frac{1}{2}$.

We investigate different covariance options in figure \ref{fig:cov_options}

\begin{figure}[btp]
	\centering
	\includegraphics[width=0.8\textwidth]{cov_options}
	\caption{The figures contain different options for $\Cov(\gamma_0, \gamma_1)$ and for all but the last graph, $\Var(\gamma_0) = \Var(\gamma_1) = 1$} 
	\label{fig:cov_options}
\end{figure}

\subsection*{Case 4}

In this case, we do the same thing as case 3 but apply it to generalize case 1. recall in case 1 we had
\begin{align*}
	\beta_i &=  (1-x)\gamma_0 + x \gamma_1 \\
	\beta_{ATE} &=  (1 - \bar{x})\gamma_0 + \bar{x} \gamma_1 \\
\end{align*}

but allowing for arbitrary correlation betweem $\gamma_0$ and $\gamma_1$.

First, we can see that when we let  $\Cov(\gamma_0, \gamma_1) = 0$, then we recover case 1.
% But if we allow $\Cov(\gamma_0, \gamma_1) = - \Var(\gamma_0)$

Furthermore, our intuition from the previous results suggest that the representation depends on the respective variances in outcomes of each group.

We will thus consider a set of options.

Given the setup, we can observe that
\begin{align*}
	\Var(\beta_{women}) &= \Var(\gamma_0) \\
	\Var(\beta_{men}) &= \Var(\gamma_1)
\end{align*} 

And in this setup, the covariance structure doesn't tell us which group has more variance, but it tells us how much learning about men tells us about the distribution of results for women and vice versa. 
With this interpretation, we can see that as $\Cov(\gamma_0, \gamma_1) \to 1$ the optimal study proportions $\bar{x} \to p$.

Likewise, as  $\Cov(\gamma_0, \gamma_1) \to 0$, we recover the results from case 1 where the equal variances tell us that because no group has greater variance, whichever group is more common in the true population should be maximized in the study.  e.g. if $p >  \frac{1}{2}$ then $\bar{x} = 1$ and vice versa.

Finally, if $\Cov(\gamma_0, \gamma_1) \in (-1,0)$ then we get edge cases where if $p > \frac{1}{2}$ then MSE is minimized at $\bar{x} = 1$ and vice versa. If $p=\frac{1}{2}$ then $\bar{x} \in \{0,1\}$ are both solutions. And as $\Cov(\gamma_0, \gamma_1) \to -1$, MSE converges pointwise to 0 with a discontinuity at $\frac{1}{2}$.

We can see these results in figure \ref{fig:generalized_case1}

\begin{figure}[ht!]
  \centering
	\includegraphics[width=0.8\textwidth]{generalized_case1}
  % \caption{Plot of $\frac{\partial J}{\partial \bar{x}}$}
	\caption{Here we show the optimal trial proportion of men vs the population proprtion (on the y and x axis respectively) for equal variances of 1.} 
	\label{fig:generalized_case1}
\end{figure}


\section{Appendix}

\begin{center}
	{\large \bf Derivations }
\end{center}


\subsection*{1. symmetric case}

For this case, let $\beta_i = (1 - x) \gamma_0 + x \gamma_1$ and define $\beta_{ATE}$ likewise.

Recall we want to minimize
\begin{align*}
\min \E_x \left(\E_{\gamma_{0,1}}[(\beta_i - \beta_{i,post})^2]\right)
\end{align*}

One way we can rewrite these equations is as the effect of women vs men.
let the squared error (SE) be
\begin{align*}
	\sqrt{SE} &= \beta_i -  \beta_{i,post} = \underbrace{[(1-x) \gamma_0 + x \gamma_1]}_{\text{$\beta_i$}}  - \underbrace{[((\vec{x} - c \vec{\bar{x}})\E(\gamma) + c \beta_{ATE})]}_\text{$\beta_{i,post}$} \\
	  &= [(1-x) \gamma_0 + x \gamma_1]  - [((1-x) - c (1 - \bar{x})) \E(\gamma_0) + (x - c \bar{x})\E(\gamma_1) + c((1-\bar{x}) \gamma_0 + \bar{x} \gamma_1)] \\
	  % &= [(1-x) \gamma_0 + x \gamma_1  - (x - c \bar{x})\E(\gamma) + c((1-\bar{x}) \gamma_0 + \bar{x} \gamma_1)] \\
\end{align*}

Because this is a function of $x$, We can then describe

$\beta_{i}^{men} = \beta_i (x=1)$ and similarly for other terms to get

So 
\begin{align*}
	\beta_i^{men} &= \gamma_1 \\
	\beta_i^{post,men} &= -c_{men} ( 1- \bar{x}) \bar{\gamma_0} + (1 - c_{men}\bar{x}) \bar{\gamma_1} + c_{men} [ (1-\bar{x}) \gamma_0 + \bar{x}\gamma_1] \\
	\beta_i^{women} &= \gamma_0 \\
	\beta_i^{post,women} &= [ 1 - c_{wom}(1-\bar{x}) ]\E(\gamma_0) - c_{wom} \bar{x} \E(\gamma_1) + c_{wom} (1- \bar{x}) \gamma_0 + c_{wom} \bar{x} \gamma_1 \\
\end{align*} 

this can be broken down into

\begin{align*}
	\sqrt{ SE } &= (\beta_i^{men} - \beta_i^{post, men} )+ (\beta_i^{wom} - \beta_i^{post, wom}) \\
\end{align*} 

let 
 \begin{align*}
	 W^{men} &=  (\beta_i^{men} - \beta_i^{post, men}) \\
			&= (1 - \bar{x} c_{men}) (\gamma_1 - \bar{\gamma_1}) - c_{men} ( 1 - \bar{x}) (\gamma_0 - \bar{\gamma_0}) \\
			&= (c_{wom} (1 - \bar{x})) (\gamma_1 - \bar{\gamma_1}) - c_{men} ( 1 - \bar{x}) (\gamma_0 - \bar{\gamma_0}) \\
	 W^{wom} &=  (\beta_i^{wom} - \beta_i^{post, wom}) \\
			 &= [ 1 - c_{wom}(1 - \bar{x}) ] (\gamma_0 - \bar{\gamma_0}) - c_{wom} \bar{x} (\gamma_1 - \bar{\gamma_1}) \\
			 &= [ c_{men} \bar{x}] (\gamma_0 - \bar{\gamma_0}) - c_{wom} \bar{x} (\gamma_1 - \bar{\gamma_1}) \\
\end{align*} 
and recall that
\begin{align*}
	c_{i} &= \frac{(1-x)(1-\bar{x})\Var(\gamma_0) + x\bar{x}\Var(\gamma_1)}{(1-\bar{x})^2\Var(\gamma_0) + \bar{x}^2\Var(\gamma_1)} \\
		  % &= \frac{(1-x)(1-\bar{x}) + x\bar{x}}{(1-\bar{x})^2 + \bar{x}^2} 
\end{align*}

so if you are a man, then 

\begin{align*}
	c_{men} =  \frac{\bar{x}\Var(\gamma_1)}{(1-\bar{x})^2\Var(\gamma_0) + \bar{x}^2\Var(\gamma_1)}
\end{align*}

and likewise if you are a woman, then

\begin{align*}
	c_{woman} = \frac{(1-\bar{x})\Var(\gamma_0)}{(1-\bar{x})^2\Var(\gamma_0) + \bar{x}^2\Var(\gamma_1)}
\end{align*}


% and 
% \begin{align*}
% 	W_{men}^2 &= \left(\frac{(1-\bar{x})^4}{((1-\bar{x})^2 + \bar{x}^2)^2}\right) \Var(\gamma_1) +  \frac{\bar{x}^2 (1-\bar{x})^2}{((1-\bar{x})^2 + \bar{x}^2)^2} \Var(\gamma_0) \\
% 			  &= \big(\frac{(1-\bar{x})^2 \Var(\gamma_1)} {(1-\bar{x})^2 + \bar{x}^2} \\
% 	W_{women}^2 &= \left(\frac{(\bar{x})^4}{((1-\bar{x})^2 + \bar{x}^2)^2}\right) \Var(\gamma_0) + \frac{\bar{x}^2 (1-\bar{x})^2}{((1-\bar{x})^2 + \bar{x}^2)^2} \Var(\gamma_1) \\
% 				&= \frac{(\bar{x})^2 \Var(\gamma_0) }{(1-\bar{x})^2 + \bar{x}^2} \\
% \end{align*} 


% so then taking the mean ($E_{x}$) with respect to $x$, we can write the mean squared error (MSE) as
% \begin{align*}
% 	MSE &= p W_{men}^2 + (1 - p)W_{wom}^2 \\
% 	&= p \left(\frac{(1-\bar{x})^4\Var(\gamma_1) + ( \bar{x}^2 (1-\bar{x})^2 ) \Var(\gamma_0)}{((1-\bar{x})^2 + \bar{x}^2)^2}\right) + (1-p)\left(\frac{(\bar{x})^4 \Var(\gamma_0) + ( \bar{x}^2 (1-\bar{x})^2 ) \Var(\gamma_1) }{((1-\bar{x})^2 + \bar{x}^2)^2}\right) \\
% \end{align*} 

% so using the fact that $\Var(\gamma_0) = \Var(\gamma_1) = 1$ we get

% \begin{align*}
% 	 MSE &= p \left(\frac{(1-\bar{x})^4 + ( \bar{x}^2 (1-\bar{x})^2 ) }{((1-\bar{x})^2 + \bar{x}^2)^2}\right) + (1-p)\left(\frac{(\bar{x})^4  + ( \bar{x}^2 (1-\bar{x})^2 ) }{((1-\bar{x})^2 + \bar{x}^2)^2}\right) \\
% 	 MSE &= p \left(\frac{(1-\bar{x})^2 }{(1-\bar{x})^2 + \bar{x}^2}\right) + (1-p)\left(\frac{(\bar{x})^2 }{(1-\bar{x})^2 + \bar{x}^2}\right) \\
% 			 &= p \frac{(1-\bar{x})^2}{(1-\bar{x})^2 + \bar{x}^2} + (1-p) \frac{\bar{x}^2}{(1-\bar{x})^2 + \bar{x}^2}
% \end{align*} 

% In other words for individual $i$ we have that

% \begin{align*}
% 	 MSE &= p (1 - c_{men} \bar{x}) + (1-p) (1 - c_{women} (1 - \bar{x})) \\
% 	 MSE &= p \left(1 - \frac{\bar{x}^2}{(1-\bar{x})^2 + \bar{x}^2}\right) + (1-p) \left(1 - \frac{(1-\bar{x})^2}{(1-\bar{x})^2 + \bar{x}^2} \right) \\
% 	 MSE &= p \left(\frac{(1 - \bar{x})^2}{(1-\bar{x})^2 + \bar{x}^2}\right) + (1-p) \left(\frac{\bar{x}^2}{(1-\bar{x})^2 + \bar{x}^2} \right) \\
% 	 MSE &= p \left[ (1 - \bar{x})c_{wom} \right] + (1-p) \left[ \bar{x}c_{men} \right] \\
% \end{align*} 

% Interestingly this implies a FOC of

% \begin{align*}
% 	\frac{\partial MSE }{\partial \bar{x}} &= p \left(\frac{2 \bar{x} (1 - \bar{x})}{((1-\bar{x})^2 + \bar{x}^2)^2}\right) - (1-p) \left(\frac{2 \bar{x} (1 - \bar{x})}{((1-\bar{x})^2 + \bar{x}^2)^2} \right) = 0 \\
% 	\frac{\partial MSE}{\partial \bar{x}} &= (2p -1) \left(\frac{2 \bar{x} (1 - \bar{x})}{((1-\bar{x})^2 + \bar{x}^2)^2}\right) = 0
% \end{align*} 

% meaning we've retrieved our original FOC.



And, allowing for arbitrary variances, we can say that

\begin{align*}
	W_{men}^2 &= \left(\frac{(1 - \bar{x})^4}{((1-\bar{x})^2 \Var(\gamma_0) + \bar{x}^2 \Var(\gamma_1))^2}\right)\Var(\gamma_0)^2 \Var(\gamma_1)  \\
			  & -  \frac{2 (1 - \bar{x})^2[\bar{x} (1-\bar{x})]}{((1-\bar{x})^2 \Var(\gamma_0) + \bar{x}^2 \Var(\gamma_1))^2} \Var(\gamma_1) \Var(\gamma_0) \Cov(\gamma_0, \gamma_1)  \\
			  & + \frac{\bar{x}^2 (1-\bar{x})^2}{((1-\bar{x})^2 \Var(\gamma_0) + \bar{x}^2 \Var(\gamma_1))^2} \Var(\gamma_1)^2 \Var(\gamma_0) \\
			  &= \Var(\gamma_0) \Var(\gamma_1) \bigg(\frac{(1 - \bar{x})^4}{((1-\bar{x})^2 \Var(\gamma_0) + \bar{x}^2 \Var(\gamma_1))^2}\Var(\gamma_0)  \\
			  & -  \frac{2 (1 - \bar{x})^2[\bar{x} (1-\bar{x})]}{((1-\bar{x})^2 \Var(\gamma_0) + \bar{x}^2 \Var(\gamma_1))^2} \Cov(\gamma_0, \gamma_1)  \\
			  & + \frac{\bar{x}^2 (1-\bar{x})^2}{((1-\bar{x})^2 \Var(\gamma_0) + \bar{x}^2 \Var(\gamma_1))^2} \Var(\gamma_1) \bigg) \\
\end{align*} 


\begin{align*}
	W_{women}^2 &= \left(\frac{(\bar{x})^4}{((1-\bar{x})^2 \Var(\gamma_0) + \bar{x}^2 \Var(\gamma_1))^2}\right)\Var(\gamma_1)^2 \Var(\gamma_0)  \\
				& -  \frac{2 \bar{x}^2[\bar{x} (1-\bar{x})]}{((1-\bar{x})^2 \Var(\gamma_0) + \bar{x}^2 \Var(\gamma_1))^2} \Var(\gamma_1) \Var(\gamma_0) \Cov(\gamma_0, \gamma_1)  \\
				& + \frac{\bar{x}^2 (1-\bar{x})^2}{((1-\bar{x})^2 \Var(\gamma_0) + \bar{x}^2 \Var(\gamma_1))^2} \Var(\gamma_0)^2 \Var(\gamma_1) \\
				&= \Var(\gamma_0) \Var(\gamma_1) \bigg(\frac{(\bar{x})^4}{((1-\bar{x})^2 \Var(\gamma_0) + \bar{x}^2 \Var(\gamma_1))^2}\Var(\gamma_1) \\
				& -  \frac{2 \bar{x}^2[\bar{x} (1-\bar{x})]}{((1-\bar{x})^2 \Var(\gamma_0) + \bar{x}^2 \Var(\gamma_1))^2} \Cov(\gamma_0, \gamma_1)  \\
				& + \frac{\bar{x}^2 (1-\bar{x})^2}{((1-\bar{x})^2 \Var(\gamma_0) + \bar{x}^2 \Var(\gamma_1))^2} \Var(\gamma_0) \bigg) \\
\end{align*} 


So letting $\Cov(\gamma_0, \gamma_1) = 0$ we get

\begin{align*}
	W_{men}^2 &= \Var(\gamma_0) \Var(\gamma_1) \bigg(\frac{(1 - \bar{x})^2 }{(1-\bar{x})^2 \Var(\gamma_0) + \bar{x}^2 \Var(\gamma_1)} \bigg) \\
\end{align*} 

\begin{align*}
	W_{women}^2 &= \Var(\gamma_0) \Var(\gamma_1) \bigg(\frac{(\bar{x})^2 }{(1-\bar{x})^2 \Var(\gamma_0) + \bar{x}^2 \Var(\gamma_1)} \bigg) \\
\end{align*} 



Now observe that if $\alpha = \Var(\gamma_1)$ and $\beta = \Var(\gamma_0)$ then
\begin{align*}
	\alpha W_{women}^2 + \beta W_{men}^2 &= \Var(\gamma_0) \Var(\gamma_1) \big[ \frac{\Var(\gamma_1) \bar{x}^2 + \Var(\gamma_0) (1- \bar{x})^2}{(1-\bar{x})^2 \Var(\gamma_0) + \bar{x}^2 \Var(\gamma_1)} \big] \\
										 &= \Var(\gamma_0) \Var(\gamma_1)
\end{align*} 
So we can write
\begin{align*}
	\alpha W^{2}_{women} &= \Var(\gamma_0) \Var(\gamma_1) - \beta W^{2}_{men} \\
	W^{2}_{women} &= \frac{\Var(\gamma_0) \Var(\gamma_1) - \beta W^{2}_{men}}{\alpha} \\
	W^{2}_{women} &= \frac{\Var(\gamma_0) \Var(\gamma_1) - \Var(\gamma_0) W^{2}_{men}}{\Var(\gamma_1)} \\
\end{align*} 

And thus conclude

\begin{align*}
	MSE &= (1 - p) W^2_{women} + p W^2_{men} \\
	MSE &= p W^{2}_{men} + (1-p) \bigg(\frac{\Var(\gamma_0) \Var(\gamma_1) - \Var(\gamma_0) W^{2}_{men}}{\Var(\gamma_1)} \bigg) \\
	MSE &= p W^{2}_{men} + (1-p) \bigg(\frac{\Var(\gamma_0) \Var(\gamma_1) - \Var(\gamma_0) W^{2}_{men}}{\Var(\gamma_1)} \bigg) \\
	&= \frac{\bigg(p[\Var(\gamma_0) + \Var(\gamma_1)]- \Var(\gamma_0)\bigg) W^{2}_{men} + (1-p) \Var(\gamma_0) \Var(\gamma_1)}{\Var(\gamma_1)}
\end{align*} 

Thus when $p > \frac{\Var(\gamma_0)}{\Var(\gamma_0) + \Var(\gamma_1)}$ we can clearly see that MSE is minimized when $W_{men}^2$ is minimized (when $\bar{x} = 1$). And inversely when $p < \frac{\Var(\gamma_0)}{\Var(\gamma_0) + \Var(\gamma_1)}$ MSE is minimized when $W_{men}^2$ is maximaximized (when $\bar{x} = 0$).
In other words, when the proportion of men is  $p > \frac{\Var(\gamma_0)}{\Var(\gamma_0) + \Var(\gamma_1)}$ it is optimal to have only men ($\bar{x}$) in the trial, and vice versa. And when there are eqal number of men and women in the population,  $MSE$ does not depend on $W_{men}^2$ and thus has equal error of $\frac{\Var(\gamma_0)}{\Var(\gamma_0) + \Var(\gamma_1)}$ for any $\bar{x}$.

We can also solve this using the first order conditions.

\begin{align*}
	MSE = (1-p)W_{women}^2 + p W_{men}^2 &= \Var(\gamma_0) \Var(\gamma_1) \big[ \frac{(1-p)\bar{x}^2 + p(1- \bar{x})^2}{(1-\bar{x})^2 \Var(\gamma_0) + \bar{x}^2 \Var(\gamma_1)} \big]
\end{align*} 

Which gives a derivative of
\begin{align*}
	\frac{MSE}{d\bar{x}} = -\frac{ 2 \Var(\gamma_0) \Var(\gamma_1) (1 - \bar{x}) \bar{x} \big[\Var(\gamma_0) (p - 1) + \Var(\gamma_1) p\big] }{(\Var(\gamma_0) (\bar{x} - 1)^2 + \Var(\gamma_1) \bar{x}^2)^2}
\end{align*} 

Here we can see that we still have roots $\bar{x} = 0$ and $\bar{x} = 1$ and all that changes is we get a variances-weighted edge-case whenever $(\Var(\gamma_0) + \Var(\gamma_1))p - \Var(\gamma_0) = 0$

And to get which is the minizing solution we can observe the SOC 
\begin{align*}
	\frac{MSE}{d \bar{x}^2} &= -\frac{2 \Var(\gamma_0) \Var(\gamma_1) [ \Var(\gamma_0) (p - 1) + \Var(\gamma_1) p ] [ \Var(\gamma_0) (2 x + 1) (x - 1)^2 + \Var(\gamma_1) x^2 (2 x - 3) ]}{[ \Var(\gamma_0) (x - 1)^2 + \Var(\gamma_1) x^2 ]^3}
\end{align*} 

From the second order condition, we can look at the two roots. We can see that the numerator reduces to two cases. When $p > \frac{\Var(\gamma_0)}{(\Var(\gamma_0) + \Var(\gamma_1))}$, then we can observe that the numerator is positive for the root $\bar{x} = 1$. Conversely, the numerator is positive when  $\bar{x} = 0$
We can also see this outlined in the simulations below.
% insert png figure named Figure_1.png here
\begin{figure}[ht!]
  \centering
	\includegraphics[width=0.8\textwidth]{simulate-sym}
  % \caption{Plot of $\frac{\partial J}{\partial \bar{x}}$}
	\caption{First column contains the simulated error for different population proportions with the population proportion shown by the blue vertical line and point of minimum error shown with the red vertical line. The second column has the analytic error computed in the math above as well as a vertical line showing the first critical point. The third column shows the first derivative with respect to $\bar{x}$.} 
	\label{fig:simulate-sym}
\end{figure}


\subsection*{2. Men Have Higher Variance}

\begin{align*}
	\beta_i &= \gamma_0 + x \gamma_1 \\
	\beta_{i}^{post} &= \E(\gamma_0) + \E(\gamma_1)  + c_i ( \gamma_0 + \bar{x} \gamma_1 - (\E(\gamma_0) + \bar{x} \E(\gamma_1))) \\
	\beta_i^{women} &= \gamma_0 \\
	\beta_i^{men} &= \gamma_0 + \gamma_1 \\
	\beta_i^{post,men} &= \E(\gamma_0) + \E(\gamma_1)  + c_{men} ( \gamma_0 + \bar{x} \gamma_1 - (\E(\gamma_0) + \bar{x} \E(\gamma_1))) \\
	\beta_i^{post, women} &= \E(\gamma_0) +  c_{women} ( \gamma_0 + \bar{x} \gamma_1 - (\E(\gamma_0) + \bar{x} \E(\gamma_1))) \\
\end{align*} 

And we have
\begin{align*}
	C_i &= \frac{\Var(\gamma_0) + x \bar{x} \Var(\gamma_1)}{\Var(\gamma_0) + \bar{x}^2 \Var{\gamma_1)}} \\
	C_{women} &= \frac{\Var(\gamma_0)}{\Var(\gamma_0) + \bar{x}^2 \Var(\gamma_1)} \\
	C_{men} &= \frac{\Var(\gamma_0) + \bar{x} \Var(\gamma_1)}{\Var(\gamma_0) + \bar{x}^2 \Var(\gamma_1)}
\end{align*} 

\begin{align*}
	W_{women} &=  \beta_i^{women} - \beta_i^{post,women} \\
			&= (\gamma_0 - \E(\gamma_0)) - C_{men} \big[(\gamma_0 - \E(\gamma_0)) + \bar{x} ( \gamma_1 - \E(\gamma_1)) \big] \\
			&= ( 1 - c_{women}) (\gamma_0 - \E(\gamma_0)) - C_{women} \bar{x} (\gamma_1 - \E(\gamma_1)) \\
	W_{men} &=  \beta_i^{men} - \beta_i^{post,men} \\
			&= (\gamma_0 - \E(\gamma_0)) + (\gamma_1 - \E(\gamma_1) - C_{men} \big[(\gamma_0 - \E(\gamma_0)) + \bar{x} ( \gamma_1 - \E(\gamma_1)) \big] \\
			&= ( 1 - c_{men}) (\gamma_0 - \E(\gamma_0)) + (1 - C_{men} \bar{x}) (\gamma_1 - \E(\gamma_1))
\end{align*} 

Squaring
\begin{align*}
	W_{women}^2 &= (1 - c_{wom})^2 \Var(\gamma_0) - 2( 1 - c_{wom})c_{wom} \bar{x}\Cov(\gamma_0, \gamma_1) + c_{wom}^2 \bar{x} \Var(\gamma_1) \\
	W_{men}^2 &= (1 - c_{men})^2 \Var(\gamma_0) + 2( 1 - c_{men})(1 - c_{men} \bar{x})\Cov(\gamma_0, \gamma_1) + (1 - c_{men} \bar{x})^2 \Var(\gamma_1) \\
\end{align*} 

And assuming $\Cov(\gamma_0, \gamma_1) = 0$ we get

\begin{align*}
	W_{women}^2 &= (1 - c_{wom})^2 \Var(\gamma_0) + c_{wom}^2 \bar{x}^2 \Var(\gamma_1) \\
				&= \frac{\bar{x}^4 \Var(\gamma_1)^2}{(\Var(\gamma_0) + \bar{x}^2 \Var(\gamma_1))^2} \Var(\gamma_0) + \frac{\bar{x}^2 \Var(\gamma_0)^2}{(\Var(\gamma_0) + \bar{x}^2 \Var(\gamma_1))^2} \Var(\gamma_1) \\
				&= \bar{x}^2 \Var(\gamma_1) \Var(\gamma_0)\frac{\bar{x}^2 \Var(\gamma_1) + \Var(\gamma_0)}{(\Var(\gamma_0) + \bar{x}^2 \Var(\gamma_1))^2} \\
				&= \frac{\bar{x}^2 \Var(\gamma_1) \Var(\gamma_0)}{\Var(\gamma_0) + \bar{x}^2 \Var(\gamma_1)} \\
	W_{men}^2 &= (1 - c_{men})^2 \Var(\gamma_0) + (1 - c_{men} \bar{x})^2 \Var(\gamma_1) \\
	          &= \frac{\Var(\gamma_1)^2[\bar{x}^2 - \bar{x}]^2}{(\Var(\gamma_0) + \bar{x}^2 \Var(\gamma_1))^2} \Var(\gamma_0) + \frac{\Var(\gamma_0)^2[1 - \bar{x}]^2}{(\Var(\gamma_0) + \bar{x}^2 \Var(\gamma_1))^2} \Var(\gamma_1) \\
			  &= \frac{\Var(\gamma_1)^2 \bar{x}^2 [\bar{x} - 1]^2}{(\Var(\gamma_0) + \bar{x}^2 \Var(\gamma_1))^2} \Var(\gamma_0) + \frac{\Var(\gamma_0)^2[\bar{x} - 1]^2}{(\Var(\gamma_0) + \bar{x}^2 \Var(\gamma_1))^2} \Var(\gamma_1) \\
			  &= \Var(\gamma_0) \Var(\gamma_1) [\bar{x} - 1]^2 \frac{\Var(\gamma_1)\bar{x}^2 + \Var(\gamma_0)}{(\Var(\gamma_0) + \bar{x}^2 \Var(\gamma_1))^2}\\
			  &= \frac{\Var(\gamma_0) \Var(\gamma_1) [\bar{x} - 1]^2 }{\Var(\gamma_0) + \bar{x}^2 \Var(\gamma_1)}\\
\end{align*} 

So we get MSE of
\begin{align*}
	MSE &= \Var(\gamma_0) \Var(\gamma_1) \frac{(1-p) \bar{x}^2 + p[\bar{x} - 1]^2 }{\Var(\gamma_0) + \bar{x}^2 \Var(\gamma_1)}\\
\end{align*} 

Taking first order conditions we get

\begin{align*}
	\frac{d}{dx} &\frac{\Var(\gamma_0) \Var(\gamma_1) [ (1 - p) x^2 + p (x - 1)^2 ]}{\Var(\gamma_0) + \Var(\gamma_1) x^2} \\
&= 2 \Var(\gamma_0) \Var(\gamma_1) \frac{ \Var(\gamma_0) (x - p) + \Var(\gamma_1) p (x - 1) x }{(\Var(\gamma_0) + \Var(\gamma_1) x^2)^2} = 0 \\
&\iff \Var(\gamma_0) (x - p) + \Var(\gamma_1) p (x - 1) x = 0 \\
&\iff x^2 \Var(\gamma_1) p + x(\Var(\gamma_0) - \Var(\gamma_1) p) - \Var(\gamma_0) p = 0
\end{align*} 

For this we can use the quadratic formula with
\begin{enumerate}
	\item $a = \Var(\gamma_1) p$
	\item $b = \Var(\gamma_0) - \Var(\gamma_1) p$
	\item $c = - \Var(\gamma_0) p$
\end{enumerate}

giving us

\begin{align*}
	x &= \frac{-b \pm \sqrt{b^2 - 4ac}}{2a} \\
	  &= \frac{-(\Var(\gamma_0) - \Var(\gamma_1) p) \pm \sqrt{(\Var(\gamma_0) - \Var(\gamma_1) p)^2 + 4 \Var(\gamma_0) \Var(\gamma_1) p^2 }}{2 \Var(\gamma_1) p} \\
	  &= \frac{\Var(\gamma_1) p-\Var(\gamma_0)  \pm \sqrt{\Var(\gamma_0)^2 - 2\Var(\gamma_0) \Var(\gamma_1) p + \Var(\gamma_1)^2 p^2 + 4 \Var(\gamma_0) \Var(\gamma_1) p^2 }}{2 \Var(\gamma_1) p} \\
	  &= \frac{\Var(\gamma_1) p-\Var(\gamma_0)  \pm \sqrt{\Var(\gamma_0)^2 + 2\Var(\gamma_0) \Var(\gamma_1) p(2p -1) + \Var(\gamma_1)^2 p^2}}{2 \Var(\gamma_1) p} \\
\end{align*} 

We can see the errors and optimal solutions \ref{fig:v2-intercept} and \ref{fig:v2-intercept-optim}


\begin{figure}[ht!]
  \centering
	\includegraphics[width=0.9\textwidth]{v2-intercept}
  % \caption{Plot of $\frac{\partial J}{\partial \bar{x}}$}
	\caption{First column contains the simulated error for different population proportions with the population proportion shown by the blue vertical line and point of minimum error shown with the red vertical line. The second column has the analytic error computed in the math above as well as a vertical line showing the first critical point and blue line showing population proportion. The third column shows the first derivative with respect to $\bar{x}$ and where it equals zero.} 
\label{fig:v2-intercept}
\end{figure}

\begin{figure}[ht!]
  \centering
	\includegraphics[width=0.8\textwidth]{Intercept_optimal}
  % \caption{Plot of $\frac{\partial J}{\partial \bar{x}}$}
	\caption{Here we show the optimal trial proportion of men vs the population proprtion (on the y and x axis respectively) for equal variances of 1.} 
\label{fig:v2-intercept-optim}
\end{figure}

\subsection*{3. Generalization of case 1 and 2}  \label{sssec:num3}

In case 1, we solved 

\begin{align*}
	\beta_i = (1 - x) \gamma_0 + x \gamma_1 = \gamma_0 + x(\gamma_1 - \gamma_0)
\end{align*} 

Thus in effect, the constant and coefficient are correlated through $\gamma_0$. We can thus nest both cases by allowing for arbitrary correlation in case 1.


We will do this below

First for $C$ 

We have that
\begin{align*}
	c = \frac{\Cov (x \gamma, \bar{x} \gamma)}{ \Var( \bar{x} \gamma )}
\end{align*} 

so for individual i, $c$ reduces to

\begin{align*}
		c &= \frac{\Var(\gamma_0) + (x + \bar{x}) \Cov(\gamma_0, \gamma_1) + x \bar{x} \Var(\gamma_1)}{\Var(\gamma_0) + 2 \bar{x} \Cov(\gamma_0, \gamma_1) + \bar{x}^2\Var(\gamma_1)} \\
		C_{women} &= \frac{\Var(\gamma_0) + \bar{x} \Cov(\gamma_0, \gamma_1)}{\Var(\gamma_0) + 2 \bar{x} \Cov(\gamma_0, \gamma_1) + \bar{x}^2\Var(\gamma_1)} \\
	C_{men} &= \frac{\Var(\gamma_0) + (1 + \bar{x}) \Cov(\gamma_0, \gamma_1) + \bar{x} \Var(\gamma_1)}{\Var(\gamma_0) + 2 \bar{x} \Cov(\gamma_0, \gamma_1) + \bar{x}^2\Var(\gamma_1)} \\
\end{align*} 

We still have

\begin{align*}
	\beta_i &= \gamma_0 + x \gamma_1 \\
	\beta_{i}^{post} &= \E(\gamma_0) + x \E(\gamma_1)  + c_i ( \gamma_0 + \bar{x} \gamma_1 - (\E(\gamma_0) + \bar{x} \E(\gamma_1))) \\
	\beta_i^{women} &= \gamma_0 \\
	\beta_i^{men} &= \gamma_0 + \gamma_1 \\
	\beta_i^{post,men} &= \E(\gamma_0) + \E(\gamma_1)  + c_{men} ( \gamma_0 + \bar{x} \gamma_1 - (\E(\gamma_0) + \bar{x} \E(\gamma_1))) \\
	\beta_i^{post, women} &= \E(\gamma_0) +  c_{women} ( \gamma_0 + \bar{x} \gamma_1 - (\E(\gamma_0) + \bar{x} \E(\gamma_1))) \\
\end{align*} 

and

\begin{align*}
	W_{women} &=  \beta_i^{women} - \beta_i^{post,women} \\
			&= (\gamma_0 - \E(\gamma_0)) - C_{men} \big[(\gamma_0 - \E(\gamma_0)) + \bar{x} ( \gamma_1 - \E(\gamma_1)) \big] \\
			&= ( 1 - c_{women}) (\gamma_0 - \E(\gamma_0)) - C_{women} \bar{x} (\gamma_1 - \E(\gamma_1)) \\
	W_{men} &=  \beta_i^{men} - \beta_i^{post,men} \\
			&= (\gamma_0 - \E(\gamma_0)) + (\gamma_1 - \E(\gamma_1) - C_{men} \big[(\gamma_0 - \E(\gamma_0)) + \bar{x} ( \gamma_1 - \E(\gamma_1)) \big] \\
			&= ( 1 - c_{men}) (\gamma_0 - \E(\gamma_0)) + (1 - C_{men} \bar{x}) (\gamma_1 - \E(\gamma_1))
\end{align*} 

Squaring
\begin{align*}
	W_{women}^2 &= (1 - c_{wom})^2 \Var(\gamma_0) - 2( 1 - c_{wom})c_{wom} \bar{x}\Cov(\gamma_0, \gamma_1) + c_{wom}^2 \bar{x}^2 \Var(\gamma_1) \\
				&=\frac{\big(\bar{x} \Cov(\gamma_0, \gamma_1) + \bar{x}^2 \Var(\gamma_1)\big)^2}{\big( \Var(\gamma_0) + 2 \bar{x} \Cov(\gamma_0, \gamma_1) + \bar{x}^2\Var(\gamma_1)\big)^2} \Var(\gamma_0) -  \\
				&2 \bigg(\frac{[ \bar{x} \Cov(\gamma_0, \gamma_1) + \bar{x}^2 \Var(\gamma_1) ][ \Var(\gamma_0) + \bar{x} \Cov(\gamma_0, \gamma_1) ]}{\big( \Var(\gamma_0) + 2 \bar{x} \Cov(\gamma_0, \gamma_1) + \bar{x}^2\Var(\gamma_1)\big)^2} \bigg) \bar{x}\Cov(\gamma_0, \gamma_1) + \\
				&\frac{\big(\Var(\gamma_0) + \bar{x} \Cov(\gamma_0, \gamma_1) \big)^2}{\big( \Var(\gamma_0) + 2 \bar{x} \Cov(\gamma_0, \gamma_1) + \bar{x}^2\Var(\gamma_1)\big)^2} \bar{x}^2 \Var(\gamma_1) \\ 
				&=\frac{\bar{x}^2 \Cov(\gamma_0, \gamma_1)^2 + 2\bar{x}^3 \Cov(\gamma_0, \gamma_1) \Var(\gamma_1) + \bar{x}^4 \Var(\gamma_1)^2 }{\big( \Var(\gamma_0) + 2 \bar{x} \Cov(\gamma_0, \gamma_1) + \bar{x}^2\Var(\gamma_1)\big)^2} \Var(\gamma_0)  \\
				&- 2 \bigg(\frac{\bar{x} \Cov(\gamma_0, \gamma_1)\Var(\gamma_0) + \bar{x}^2 \Cov(\gamma_0, \gamma_1)^2 + \bar{x}^2 \Var(\gamma_0) \Var(\gamma_1)  + \bar{x}^3 \Var(\gamma_1)\Cov(\gamma_0, \gamma_1)}{\big( \Var(\gamma_0) + 2 \bar{x} \Cov(\gamma_0, \gamma_1) + \bar{x}^2\Var(\gamma_1)\big)^2} \bigg) \bar{x}\Cov(\gamma_0, \gamma_1) \\
				&+ \frac{\Var(\gamma_0)^2 + 2 \bar{x} \Var(\gamma_0) \Cov(\gamma_0, \gamma_1) + \bar{x}^2 \Cov(\gamma_0, \gamma_1)^2 }{\big( \Var(\gamma_0) + 2 \bar{x} \Cov(\gamma_0, \gamma_1) + \bar{x}^2\Var(\gamma_1)\big)^2} \bar{x}^2 \Var(\gamma_1) \\ 
				  &= \frac{\bar{x}^2 \big[ \Var(\gamma_0) \Var(\gamma_1) - \Cov(\gamma_0, \gamma_1)^2 \big]}{\Var(\gamma_0) + \Var(\gamma_1) \bar{x}^2 + 2 \Cov(\gamma_0, \gamma_1) \bar{x}}
\end{align*} 

\begin{align*}
	W_{men}^2 &= (1 - c_{men})^2 \Var(\gamma_0) + 2( 1 - c_{men})(1 - c_{men} \bar{x})\Cov(\gamma_0, \gamma_1) + (1 - c_{men} \bar{x})^2 \Var(\gamma_1) \\
			  &= \frac{\big((\bar{x} - 1) \Cov(\gamma_0, \gamma_1) + (\bar{x}^2 - \bar{x}) \Var(\gamma_1)\big)^2}{\big(\Var(\gamma_0) + 2 \bar{x} \Cov(\gamma_0, \gamma_1) + \bar{x}^2\Var(\gamma_1)\big)^2} \Var(\gamma_0) \\
			  &+ 2\frac{[(\bar{x} - 1) \Cov(\gamma_0, \gamma_1) + (\bar{x}^2 - \bar{x}) \Var(\gamma_1)][(1 - \bar{x}) \Var(\gamma_0) + (\bar{x} - \bar{x}^2) \Cov(\gamma_0, \gamma_1) ]}{\big(\Var(\gamma_0) + 2 \bar{x} \Cov(\gamma_0, \gamma_1) + \bar{x}^2\Var(\gamma_1) \big)^2}\Cov(\gamma_0, \gamma_1) \\
			  &+ \frac{[(1 - \bar{x}) \Var(\gamma_0) + (\bar{x} - \bar{x}^2) \Cov(\gamma_0, \gamma_1)]^2}{\big(\Var(\gamma_0) + 2 \bar{x} \Cov(\gamma_0, \gamma_1) + \bar{x}^2\Var(\gamma_1) \big)^2} \Var(\gamma_1) \\
			  &= \frac{(\bar{x} - 1)^2 \big[ \Var(\gamma_0) \Var(\gamma_1) - \Cov(\gamma_0, \gamma_1)^2 \big]}{\Var(\gamma_0) + \Var(\gamma_1) \bar{x}^2 + 2 \Cov(\gamma_0, \gamma_1) \bar{x}}
\end{align*} 

We can see then that if $\Cov(\gamma_0, \gamma_1) = 0$ we recover our result from version 2.


And we get a mean square error of 
\begin{align*}
	MSE = \big(\Var(\gamma_0) \Var(\gamma_1) - \Cov(\gamma_0, \gamma_1)^2 \big)\frac{(1 - p) \bar{x}^2  + p(\bar{x} - 1)^2 }{\Var(\gamma_0) + \Var(\gamma_1) \bar{x}^2 + 2 \Cov(\gamma_0, \gamma_1) \bar{x}}
\end{align*} 

\begin{align*}
	\frac{d}{dx} &\big(\Var(\gamma_0) \Var(\gamma_1) - \Cov(\gamma_0, \gamma_1)^2\big)\frac{(1 - p) \bar{x}^2 + p (\bar{x} - 1)^2}{ \Var(\gamma_0) + \Var(\gamma_1) \bar{x}^2 + 2 \bar{x} \Cov(\gamma_0, \gamma_1) } \\
	&=2 \big(\Var(\gamma_0) \Var(\gamma_1) - \Cov(\gamma_0, \gamma_1)^2 \big) \frac{(\Var(\gamma_0) (\bar{x} - p) + \Var(\gamma_1) p (\bar{x} - 1) \bar{x} + \Cov(\gamma_0, \gamma_1) (\bar{x}^2 - p)) }{(\Var(\gamma_0) + \bar{x} (\Var(\gamma_1) \bar{x} + 2 \Cov(\gamma_0, \gamma_1)))^2}
\end{align*} 

Thus we can use the quadratic formula to determine the optimal trial proportions.

\begin{align*}
	\bar{x} = \frac{\Var(\gamma_1) p - \Var(\gamma_0) \pm \sqrt{ 4 p \big[ \Var(\gamma_0) + \Cov(\gamma_0, \gamma_1) \big] \big[ \Var(\gamma_1) p + \Cov(\gamma_0, \gamma_1) \big] + \big(\Var(\gamma_0) - \Var(\gamma_1) p\big)^2 } }{ 2 (\Var(\gamma_1) p + \Cov(\gamma_0, \gamma_1)) } 
\end{align*} 
and clearly $\Var(\gamma_1) p + \Cov(\gamma_0, \gamma_1) \neq 0$


We can thus see that we recover our original case of choosing between all men and all women whenever
\begin{align*}
	\Cov(\gamma_0, \gamma_1) = - \Var(\gamma_0)
\end{align*} 

And when this holds, when $p = \frac{\Var(\gamma_0)}{\Var(\gamma_1)}$, we again get our edge-case where the the error is independent of the trial proportions and any $\bar{x}$ produces an equal error

Now we can check whether our setup in version 1 satisfies this.

Recall we have
\begin{align*}
	\beta_i = \gamma_0 + x ( \gamma_1 - \gamma_0)
\end{align*} 

Let's call $\gamma_2 = \gamma_1 - \gamma_0$

First observe that 

\begin{align*}
	\Var(\gamma_2) &= \Var(\gamma_1 - \gamma_0) = \Var(\gamma_0) + \Var(\gamma_1)  - 2 \Cov(\gamma_0, \gamma_1) \quad \text{assuming both are 1} \\
				   &= 1 + 1 - 0 = 2
\end{align*} 

and 
\begin{align*}
	\Cov(\gamma_0, \gamma_2)	&= \Cov(\gamma_0, \gamma_1 - \gamma_0) = \Cov(\gamma_0, \gamma_1) - \Cov(\gamma_0, \gamma_0) \\
										&= 0 - 1 = -1
\end{align*} 

And so we recover that
\begin{align*}
	\Cov(\gamma_0, \gamma_2) &= -\Var(\gamma_{0})
\end{align*} 

And more over that
\begin{align*}
	p = \frac{\Var(\gamma_0)}{\Var(\gamma_2)} &= \frac{1}{2}
\end{align*} 

And thus and MSE

\begin{align*}
	MSE &= \big(\Var(\gamma_0) \Var(\gamma_1) - \Cov(\gamma_0, \gamma_1)^2 \big)\frac{(1 - p) \bar{x}^2  + p(\bar{x} - 1)^2 }{\Var(\gamma_0) + \Var(\gamma_1) \bar{x}^2 + 2 \Cov(\gamma_0, \gamma_1) \bar{x}} \\
		&= (2 - 1)\frac{\frac{1}{2} \bar{x}^2 + \frac{1}{2}(\bar{x}^2 - 2 \bar{x} + 1)}{1 - 2 \bar{x} + 2 \bar{x}^2} \\
		&= \frac{\frac{1}{2} \bar{x}^2 + \frac{1}{2}(\bar{x}^2 - 2\bar{x} + 1)}{2(\frac{1}{2} -  \bar{x} + \bar{x}^2)} = \frac{1}{2}
\end{align*} 

Which is always equal to $\frac{1}{2}$ and independent of $\bar{x}$. This is precisely what we get with the original solution to case 1.

\subsection*{4. Generalization of case 1 and 2} \label{sssec:num4}

In case 1, we solved 

\begin{align*}
	\beta_i = (1 - x) \gamma_0 + x \gamma_1 = \gamma_0 + x(\gamma_1 - \gamma_0)
\end{align*} 

Thus in effect, the constant and coefficient are correlated through $\gamma_0$. We can thus nest both cases by allowing for arbitrary correlation in case 1.


We will do this below

First for $C$ 

We have that
\begin{align*}
	c = \frac{\Cov (x \gamma, \bar{x} \gamma)}{ \Var( \bar{x} \gamma )}
\end{align*} 

so for individual i, $c$ reduces to

\begin{align*}
	c &= \frac{(1-x)(1 - \bar{x})\Var(\gamma_0) + x (1 - \bar{x}) \Cov(\gamma_0, \gamma_1) +(1 - x)\bar{x}  \Cov(\gamma_0, \gamma_1) +   x \bar{x} \Var(\gamma_1)}{(1- \bar{x})^2 \Var(\gamma_0) + 2 \bar{x}(1-\bar{x}) \Cov(\gamma_0, \gamma_1) + \bar{x}^2\Var(\gamma_1)} \\
	C_{women} &= \frac{(1 - \bar{x})\Var(\gamma_0) + \bar{x} \Cov(\gamma_0, \gamma_1)}{(1-\bar{x})^2\Var(\gamma_0) + 2 (1 - \bar{x})\bar{x} \Cov(\gamma_0, \gamma_1) + \bar{x}^2\Var(\gamma_1)} \\
	C_{men} &= \frac{(1 - \bar{x}) \Cov(\gamma_0, \gamma_1) + \bar{x} \Var(\gamma_1)}{(1-\bar{x})^2\Var(\gamma_0) + 2 (1 - \bar{x})\bar{x} \Cov(\gamma_0, \gamma_1) + \bar{x}^2\Var(\gamma_1)} \\
\end{align*} 

We still have

\begin{align*}
	\beta_i &= (1 -x)\gamma_0 + x \gamma_1 \\
	\beta_{i}^{post} &= (1-x)\E(\gamma_0) + x\E(\gamma_1)  + c_i [ (1-\bar{x})\gamma_0 + \bar{x} \gamma_1 - [ (1- \bar{x})\E(\gamma_0) + \bar{x} \E(\gamma_1) ] ] \\
	\beta_i^{women} &= \gamma_0 \\
	\beta_i^{men} &=  \gamma_1 \\
	\beta_i^{post,men} &=  \E(\gamma_1)  + c_{men} ((1-\bar{x})\gamma_0 + \bar{x} \gamma_1 - ((1-\bar{x})\E(\gamma_0) + \bar{x} \E(\gamma_1))) \\
	\beta_i^{post, women} &= \E(\gamma_0) +  c_{women} ((1-\bar{x}) \gamma_0 + \bar{x} \gamma_1 - ((1-\bar{x})\E(\gamma_0) + \bar{x} \E(\gamma_1))) \\
\end{align*} 

and

\begin{align*}
	W_{women} &=  \beta_i^{women} - \beta_i^{post,women} \\
			&= (\gamma_0 - \E(\gamma_0)) - C_{men} \big[(\gamma_0 - \E(\gamma_0)) + \bar{x} ( \gamma_1 - \E(\gamma_1)) \big] \\
			&= ( 1 - c_{women}(1-\bar{x})) (\gamma_0 - \E(\gamma_0)) - C_{women} \bar{x} (\gamma_1 - \E(\gamma_1)) \\
			&= c_{men}\bar{x} (\gamma_0 - \E(\gamma_0)) - C_{women} \bar{x} (\gamma_1 - \E(\gamma_1)) \\
	W_{men} &=  \beta_i^{men} - \beta_i^{post,men} \\
			&= (\gamma_0 - \E(\gamma_0)) + (\gamma_1 - \E(\gamma_1) - C_{men} \big[(\gamma_0 - \E(\gamma_0)) + \bar{x} ( \gamma_1 - \E(\gamma_1)) \big] \\
			&= ( 1 - c_{men} \bar{x}) (\gamma_1 - \E(\gamma_1)) - C_{men} (1-\bar{x}) (\gamma_0 - \E(\gamma_0)) \\
			&=  c_{women} (1 - \bar{x}) (\gamma_1 - \E(\gamma_1)) - C_{men} (1-\bar{x}) (\gamma_0 - \E(\gamma_0))
\end{align*} 

\begin{align*}
	W_{women}^2 &= (C_{men} \bar{x})^2\Var(\gamma_0)\\
				& -  2 \bar{x} ^2 C_{women} C_{men}\Cov(\gamma_0, \gamma_1)  \\
				& + (\bar{x} C_{women})^2\Var(\gamma_1) \\
				&= \frac{ (\Var(\gamma_0) \Var(\gamma_1) - \Cov(\gamma_0, \gamma_1)^2) \bar{x}^2 }{ \bar{x}^2 (\Var(\gamma_0) + \Var(\gamma_1) - 2 \Cov(\gamma_0, \gamma_1)) + 2 \bar{x} (\Cov(\gamma_0, \gamma_1) - \Var(\gamma_0)) + \Var(\gamma_0) }
\end{align*} 


\begin{align*}
	W_{men}^2 &= (C_{women} (1-\bar{x}))^2\Var(\gamma_1)\\
			  & -  2 (1-\bar{x})^2 C_{women} C_{men}\Cov(\gamma_0, \gamma_1)  \\
				& + ((1-\bar{x}) C_{men})^2\Var(\gamma_0) \\
				&= \frac{ (\bar{x} - 1)^2 (\Var(\gamma_0) \Var(\gamma_1) - \Cov(\gamma_0, \gamma_1)^2) }{ \bar{x}^2 (\Var(\gamma_0) + \Var(\gamma_1) - 2 \Cov(\gamma_0, \gamma_1)) + 2 \bar{x} (\Cov(\gamma_0, \gamma_1) - \Var(\gamma_0)) + \Var(\gamma_0) }
\end{align*} 

And so we get
\begin{align*}
	MSE &= (1-p) W_{women}^2 + p W_{men}^2 \\
		&=  \frac{(\Var(\gamma_0) \Var(\gamma_1) - \Cov(\gamma_0, \gamma_1)^2)[(1-p) \bar{x}^2   + p (\bar{x} - 1)^2 ]} { \bar{x}^2 (\Var(\gamma_0) + \Var(\gamma_1) - 2 \Cov(\gamma_0, \gamma_1)) + 2 \bar{x} (\Cov(\gamma_0, \gamma_1) - \Var(\gamma_0)) + \Var(\gamma_0) }
\end{align*} 

and derivative
\begin{scriptsize}
\begin{align*}
	\frac{d MSE}{d\bar{x}} &= 
	\frac{ 2 (\Var(\gamma_0) \Var(\gamma_1) - \Cov(\gamma_0, \gamma_1)^2) ((\bar{x} - 1) \bar{x} (\Var(\gamma_0) (p - 1) + \Var(\gamma_1) p) + \Cov(\gamma_0, \gamma_1) (p (-2 (\bar{x} - 1) \bar{x} - 1) + \bar{x}^2)) }{(\bar{x}^2 (\Var(\gamma_0) + \Var(\gamma_1) - 2 \Cov(\gamma_0, \gamma_1)) + 2 \bar{x} (\Cov(\gamma_0, \gamma_1) - \Var(\gamma_0)) + \Var(\gamma_0))^2}
\end{align*} 
\end{scriptsize}


Thus taking the numerator and setting it equal to zero we get an FOC

\begin{tiny}
\begin{align*}
	\bar{x} = \frac{ \Var(\gamma_0) (p - 1) + \Var(\gamma_1) p - 2 \Cov(\gamma_0, \gamma_1) p \pm \sqrt{ \Var(\gamma_0)^2 (p - 1)^2 + 2 \Var(\gamma_0) \Var(\gamma_1) p (p - 1) + p (\Var(\gamma_1)^2 p - 4 \Cov(\gamma_0, \gamma_1)^2 (p - 1)) } }{ 2 (\Var(\gamma_0) (p - 1) + \Var(\gamma_1) p - 2 \Cov(\gamma_0, \gamma_1) p + \Cov(\gamma_0, \gamma_1)) }
\end{align*} 
\end{tiny}


We can see from the above solution that whenever
\begin{align*}
	\Var(\gamma_0) = 2 \Cov(\gamma_0, \gamma_1) p = p (\Var(\gamma_0) + \Var(\gamma_1)) + \Cov(\gamma_0, \gamma_1)
\end{align*} 
That the solution is undefined. In other words, when $\Var(\gamma_0) = \Var(\gamma_1) = \Cov(\gamma_0, \gamma_1) = 1$ there is no solution. 

This makes sense because if the results are perfectly correlated and no group has higher variance, than any choice of representation should tell us the exact same amount about the other group. And thus there is no clear optimal solution.

\subsection*{5. Generalization of case 2 allowing for $k$ subgroups.}

\begin{align*}
	\beta_i &= (1-x_1)(1- x_2) \gamma_0 + (1-x_1) x_2 \gamma_1 +  x_1(1- x_2) \gamma_2 +  x_1 x_2 \gamma_3 \\
	\beta_{ATE} &= (1-\bar{x}_1)(1- \bar{x}_2) \gamma_0 + (1-\bar{x}_1) \bar{x}_2 \gamma_1 +  \bar{x}_1(1- \bar{x}_2) \gamma_2 +  \bar{x}_1 \bar{x}_2 \gamma_3 \\
\end{align*} 

Observe that for $k$ subgroups, this produces $2^{k}$ interactions. In other words, we are simple enumerating the binary expansion for all possible combinations of $k$ characteristics.


We will start by assuming $\Cov(\gamma_i, \gamma_j) = 0$ for $i \neq j$ and  $\Var(\gamma_i) = 1$ for every $i$.

In addition we will start by solving the case for $k = 2$ and then attempting to generalize.

We will do this below

First, for simplicity, let $ \alpha_i$ be the ith subgroup created by our k characteristics. In other words, $\alpha_0 = (1-x_1) (1 - x_2)$ and $\bar{\alpha}_0 = (1-\bar{x}_1)(1 - \bar{x}_2)$ and 
$\alpha_5 = x_1 x_2$ with $\bar{\alpha} $ defined similarly.

And more over, let $A$ be a vector such that
\begin{align*}
	A = \begin{bmatrix}
		\alpha_0 \\
		\alpha_1 \\
		\vdots \\
		\alpha_{2^{k}-1}
	\end{bmatrix}
\end{align*}

and $\bar{A}$ be define likewise.

Then For $C$ we have that
\begin{align*}
	c &= \frac{\Cov (\beta_i, \beta_{ATE})}{ \Var(\beta_{ATE})} \\
	  &= \frac{\sum_{i=0}^{2^{k}-1} \alpha_i \bar{\alpha}_i \Var(\gamma_i)}{\sum_{i=0}^{2^{k}-1} ( \bar{\alpha}_i)^2 \Var(\gamma_i)} \qquad \text{assuming $\Var(\gamma_i) = 0$ } \\
	  &= \frac{\sum_{i=0}^{2^{k}-1} \alpha_i \bar{\alpha}_i }{\sum_{i=0}^{2^{k}-1} ( \bar{\alpha}_i)^2 } \\ 
	  &= \frac{A' \bar{A}}{\bar{A}' \bar{A}} \\
\end{align*} 

so for individual i, $c$ reduces to

\begin{align*}
	c &= \frac{(1-x_1)(1- x_2)(1- \bar{x}_1)(1-\bar{x}_2) \Var(\gamma_0) +x_2 \bar{x}_2 \Var(\gamma_2)}{\Var(\gamma_0) + 2 \bar{x} \Cov(\gamma_0, \gamma_1) + (1- \bar{x}_1) (1 - \bar{x}_2)\Var(\gamma_3)} \\
	  &= \frac{\sum_{i=1}^{k} (x_i \bar{x}_i) \Var(\gamma_i)  } {}\\
\end{align*} 


For now, let $k=2$

and assume $\Cov(\gamma_i, \gamma_j) = 0$ for $i \neq j$

Let $x_1 = $ white, $x_2=$ black

\begin{align*}
		C_{white} &= \frac{\Var(\gamma_0) + \bar{x}_1 \Var(\gamma_1) }{\Var(\gamma_0) + \bar{x}_1^2\Var(\gamma_1) + \bar{x}_2^2\Var(\gamma_2)} \\
	C_{black} &= \frac{\Var(\gamma_0) + \bar{x}_2 \Var(\gamma_2)}{\Var(\gamma_0) + \bar{x}_1^2\Var(\gamma_1) + \bar{x}_2^2\Var(\gamma_2)} \\
\end{align*} 


we want to minimize

\begin{align*}
	\E [(\beta_i - \beta_{i, post})^2]
\end{align*} 

\begin{align*}
	\E((x_i - c_i \bar{x}) \gamma | \bar{x} \gamma = \beta_{ATE}) = (x_{i} - c_i \bar{x}) \E(\gamma)
\end{align*}
where, for the $k = 1$ case,
\begin{align*}
	\beta_{i, post} = (x_i - c_i \bar{x}) \E(\gamma) + c_i \beta_{ATE} = x_i \E(\gamma) + c_i (\beta_{ATE} - \bar{x} \E(\gamma))
\end{align*} 

rewriting this in the matrix form, we have that

\begin{align*}
	\beta_i = A' \gamma
\end{align*} 

where $\gamma$ is the vector of $\gamma_i$ e.g.

\begin{align*}
	\gamma = \begin{bmatrix}
		\gamma_0 \\
		\gamma_1 \\
		\ldots \\
		\gamma_{2^{k}-1}
	\end{bmatrix}
\end{align*}


then we get 
 \begin{align*}
	\beta_{ATE} = \bar{A}' \gamma
\end{align*} 

\begin{align*}
	\beta_{i, post} = A' \E(\gamma) + \frac{A' \bar{A}}{\bar{A}' \bar{A}} (\bar{A}' \gamma - \bar{A}' \E(\gamma))
\end{align*} 

then our error becomes

\begin{align*}
	(\beta_{i} - \beta_{i, post})^2
	&= (A'\gamma - \big[A' \E(\gamma) + \frac{A' \bar{A}}{\bar{A}' \bar{A}} (\bar{A}' \gamma - \bar{A}' \E(\gamma))\big])^2 \\
	&= \big[A'(\gamma - \E(\gamma)) - \frac{A' \bar{A}}{\bar{A}' \bar{A}} \bar{A}' (\gamma - \E(\gamma))\big]^2
\end{align*} 

function:
\[
  f = (A^\top \cdot (g-f)-(A^\top \cdot B)/(B^\top \cdot B)\cdot B^\top \cdot (g-f))^{2}
\]

gradient:

% \[
%   \frac{\partial f}{\partial B} = -((2\cdot ((g-f)' \cdot A-(B' \cdot A\cdot (g-f)' \cdot B)/(B' \cdot B)))/(B' \cdot B)\cdot (g-f)' \cdot B\cdot A+(2\cdot (A' \cdot (g-f)-(A' \cdot B\cdot B' \cdot (g-f))/(B' \cdot B)))/(B' \cdot B)\cdot A' \cdot B\cdot (g-f)-((2\cdot (A' \cdot (g-f)-(A' \cdot B\cdot B' \cdot (g-f))/(B' \cdot B)))/(B' \cdot B)^{2}\cdot A' \cdot B\cdot B' \cdot (g-f)\cdot B+(2\cdot ((g-f)' \cdot A-(B' \cdot A\cdot (g-f)' \cdot B)/(B' \cdot B)))/(B' \cdot B)^{2}\cdot B' \cdot A\cdot (g-f)' \cdot B\cdot B))
% \]

\begin{align*}
	&\frac{\partial f}{\partial \bar{A}} = -(\frac{(2 ((\gamma-\E(\gamma))'  A-\frac{(\bar{A}'  A (\gamma-\E(\gamma))'  \bar{A})}{(\bar{A}'  \bar{A})}))}{(\bar{A}'  \bar{A})} (\gamma-\E(\gamma))'  \bar{A} A \\
	&+ \frac{(2 (A'  (\gamma-\E(\gamma))-\frac{(A'  \bar{A} \bar{A}'  (\gamma-\E(\gamma)))}{(\bar{A}'  \bar{A})}))}{(\bar{A}'  \bar{A})} A'  \bar{A} (\gamma-\E(\gamma)) \\
	&-(\frac{(2 (A'  (\gamma-\E(\gamma))-\frac{(A'  \bar{A} \bar{A}'  (\gamma-\E(\gamma)))}{(\bar{A}'  \bar{A})}))}{(\bar{A}'  \bar{A})^{2}} A'  \bar{A} \bar{A}'  (\gamma-\E(\gamma)) \bar{A} \\
	&+\frac{(2 ((\gamma-\E(\gamma))'  A-\frac{(\bar{A}'  A (\gamma-\E(\gamma))'  \bar{A})}{(\bar{A}'  \bar{A})}))}{(\bar{A}'  \bar{A})^{2}} \bar{A}'  A (\gamma-\E(\gamma))'  \bar{A} \bar{A}))
\end{align*} 

Now note that $\frac{\partial}{\partial x} E( f(A, \gamma, \bar{A}) ) = E(\frac{\partial}{\partial x} f(A, \gamma, \bar{A}))$ (I believe this just follows from linearity of expectation but we likely need bounded and continuously differentiable but those should be satisfied).



\end{document}

